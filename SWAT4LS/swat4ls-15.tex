
\documentclass[runningheads,a4paper]{llncs}

\usepackage{amssymb}
\setcounter{tocdepth}{3}
\usepackage{graphicx}

\begin{document}

% first the title is needed
\title{COPO - Linked Open Infrastructure for Plant Data}

\author{F Shaw \inst{1} \and A Etuk \inst{1} \and A Gonzalez-Beltran
  \inst{2}\and P Rocca-Serra \inst{2}\and\\
  P  Kersey \inst{3}\and R Bastow \inst{4} \and S Sansone \inst{2}\and\\
  V Schneider \inst{1}\and R Davey \inst{1}}
%

\institute{The Genome Analysis Centre, UK \and Oxford
  e-Research Centre, University of Oxford, UK \and  European
  Bioinformatics Institute, Cambridge \and University of Warwick}

\maketitle

\vspace*{-0.2in}
\begin{abstract}
  We present Collaborative Open Plant Omics (COPO), a brokering
  service between plant scientists and public repositories, which
  enables management, aggregation and publication of research
  outputs. COPO provides consolidated access to services and disparate
  information sources via web interfaces and Application Programming
  Interfaces (APIs). Users will be able to deposit and view open
  access data, as well as seamlessly pull such data into suitable
  analysis environments and subsequently track the outputs and
  associated metadata in COPO, thus creating a provenance trail from
  data to publication.
\end{abstract}

\vspace*{-0.3in}
\section{Introduction}
\vspace*{-0.1in}
The plant science domain has seen the advent of increasingly high
throughput ?-omics? technologies, resulting in more and larger
datasets being produced. Researchers are realising the benefits of
data sharing to promote their work and to accelerate discovery in
science based on aggregated data. Many funding bodies and journals now
require that data be made publicly available. Despite the
opportunities that data sharing offers for recognition and reuse, many
scientists still do not use public repositories, choosing instead to
store data privately in their organisation?s infrastructure. The
reasons for this include lack of awareness of where and how to deposit
data, lack of standards and common metadata, and a lack of funding to
support archiving.

The large number and size of the datasets make them difficult to
store, let alone download, making cloud-based analysis software highly
desirable. However, submission formats to public repositories are
heterogeneous, often requiring manual authoring of complex markup
documents, taking scientists out of their fields of expertise.

COPO aims to streamline the process of data deposition to public
repositories and data journals, by hiding much of the complexity of
metadata capture and data management from the end-user. The ISA
infrastructure (www.isa-tools.org) is leveraged to provide the
interoperability between metadata formats required for seamless
deposition to repositories and to facilitate links to data analysis
platforms. Logical groupings of artefacts (e.g. experimental metadata
and results, PDFs, raw data, contextual supplementary information)
relating to a body of work are stored in COPO ``collections'' and
represented by common open standards, which are publicly
searchable. Bundles of multiple data objects themselves can then be
deposited directly into public repositories through COPO interfaces.

\vspace*{-0.2in}
\section{Core Features}
\vspace*{-0.1in}
\begin{itemize}
\item User Interfaces
  \begin{itemize}
  \item Web-based tools enable consolidated access to a range of data
    repositories
  \item Allows for intuitive and intelligent labelling of data
    according to accepted standards
  \end{itemize}
\item Data Deposition, Querying and Publication
  \begin{itemize}
  \item APIs enable deposition of data and metadata to public
    repositories such as the European Nucleotide Archive and Figshare
  \item APIs allow querying of metadata and access to research
    artefacts deposited in public archives
  \item APIs to allow submission of data and metadata to publication
    platforms such as Scientific Data and F1000
  \item Prototype wizards to help users through the metadata
    attribution process
  \end{itemize}
\end{itemize}

\vspace*{-0.3in}
\section{Metadata Management}
\vspace*{-0.1in}
The Investigation/Study/Assay (ISA) model enables experimental
metadata attribution and management of metadata formats, where
scientific metadata comprises information about investigators,
objectives, hypotheses, publications, subjects, experimental design,
experimental workflow, and assays and related experimental data. ISA
metadata is represented in ISA-JSON, and integrated within a broader
subset of metadata, COPO-JSON, that encompasses infrastructural
information relative to the platform itself. Both JSON implementations
can be extended to JSON-LD linked data schemas. All JSON metadata
fragments are stored in a MongoDB document-based database.

Where required, ISA converters allow traversal between representations
of the same metadata, e.g. ISATab to/from ISA-JSON, and public
repository formats are expressed as ISA configurations which are
mapped to a COPO-JSON user interface (UI) model to power the COPO UI
itself. In this way, we can quickly and easily adapt to new
repositories or changes to existing repository schemas all the way
from data representation to UI design.

\vspace*{-0.1in}
\section{Platform in Development}
\vspace*{-0.1in}
The COPO framework is being built using Python, Django, MongoDB,
JSON-LD, ISATools, jQuery and Bootstrap technologies. We provide a
single sign-on (SSO) mechanism via ORCiD which allows COPO to track
service integration and rich user profile data. Anonymous users are
able to search the COPO index for research artefacts, whereas
deposition functionality is available to authenticated users only. The
complexity of deposition services is hidden from end users, who simply
fill out clean, intuitive web forms and story-driven wizards that use
the semantic level metadata to make inferences about what a user is
submitting, and subsequently make suggestions based on previous
submissions.

So far we have developed initial EMBL-EBI repository deposition
support (European Nucleotide Archive (ENA), MetaboLights) facilitated
by Aspera-powered data transfer and ISA API integration. Figshare
deposition of secondary research artefacts (PDFs, images, figures,
supplementary data, etc) is also supported.

\vspace*{-0.1in}
\section{Future Work}
\vspace*{-0.1in}
The large network of linked metadata that COPO will gather allows
semantic meaning to be attached to research artefacts. Semantic
inferences can then be made over artefacts providing a richer search
experience than through text based search alone, enabling researchers
to quickly find and use well-described publicly available datasets
linked by a full ?paper trail? of metadata. The provision of
visualisation for graphs of linked metadata will aid discovery of
useful connections between datasets, investigations and
protocols. Support for more repositories and open publishing platforms
such as GigaScience, F1000, Scientific Data and Dryad are planned, as
well as integration with cloud-based analysis services such as Galaxy
and iPlant.


\end{document}
